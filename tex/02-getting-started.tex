\section{Getting Started}

\subsection{About Version Control}
\begin{frame}[t]{Local Version Control System}
  \begin{center}
    \includegraphics[height=2.75in]{../images/02-getting-started/local}
  \end{center}
\end{frame}

\begin{frame}[t]{Centralized Version Control Systems}
  \begin{center}
    \includegraphics[height=2.75in]{../images/02-getting-started/centralized}
  \end{center}
\end{frame}

\begin{frame}[t]{Distributed Version Control Systems}
  \begin{center}
    \includegraphics[height=2.75in]{../images/02-getting-started/distributed}
  \end{center}
\end{frame}

\subsection{A Short History of Git}
\begin{frame}[t]{Short History of Git}
  \begin{itemize}
    \item Linux (Linus Torvalds) vs BitKeeper
    \item 2005: Linux development community sets out to develop their own DVCS
      with goals for the new system of
      \begin{itemize}
        \item speed,
        \item simple design,
        \item strong support for non-linear development (thousands of parallel
          branches),
        \item fully distributed, and
        \item albe to handle large projects, like the Linux kernel,
          efficiently.
      \end{itemize}
  \end{itemize}
\end{frame}

\subsection{Git Basics}
\begin{frame}[t,allowframebreaks]{Snapshots, Not Differences}

  The Subversion model: file-based changes.

  \begin{center}
    \includegraphics[width=0.98\textwidth]{../images/02-getting-started/deltas}
  \end{center}

  \pagebreak

  The Git model: a stream of snapshots.

  \begin{center}
    \includegraphics[width=0.98\textwidth]{../images/02-getting-started/snapshots}
  \end{center}

\end{frame}

\begin{frame}[t]{Nearly Every Operation is Local}
  \begin{itemize}
    \item Most operations are local.
      \begin{itemize}
        \item No need to talk to other computers on a network.
        \item The entire project history is on your local disk.
      \end{itemize}

    \item You can work offline.
    \item You can work off VPN.  
  \end{itemize}
\end{frame}

\begin{frame}[t]{Git Has Integrity}
  \begin{itemize}
    \item Everything is check-summed (remember the sha?)
    \item You can't lose informatin in transit nor get a file corruption without
      Git being able to detect it.
    \item Git stores everything in its database not by file name but by the hash
      value of its contents.
  \end{itemize}
\end{frame}

\begin{frame}[t]{Git Generally Only Adds Data}
  \begin{itemize}
    \item Nearly all actions in Git add data to the Git database.
    \item It is difficult to do anything that is not undoable.
    \item You can lose/corupt un-committed changes.
    \item It is very difficult to lose anything after a commit, especially with
      frequent pushes to other repositories.
  \end{itemize}
\end{frame}

\begin{frame}[t]{The Three States}
  This is the main thing to remember about Git if you want the rest of your
  learning process to go smoothly.  Git has three main states that files reside
  in.

  \begin{enumerate}
    \item Committed
      \begin{itemize}
        \item data is safely stored in the local database.
      \end{itemize}
    \item Modified
      \begin{itemize}
        \item changed the file(s) but have not committed to the database yet.
      \end{itemize}
    \item Staged
      \begin{itemize}
        \item Marked modified file(s) in its current version to go into the next
          commit snapshot.
      \end{itemize}
  \end{enumerate}

  This leads to three main sections of a Git project:
  \begin{enumerate}
    \item the Git directory,
    \item the working directory, and
    \item the staging area.
  \end{enumerate}
\end{frame}

\subsection{The Command Line}
\begin{frame}[t]{}
\end{frame}

\subsection{Intalling Git}
\begin{frame}[t]{}
\end{frame}

\begin{frame}[t]{}
\end{frame}

\begin{frame}[t]{}
\end{frame}

\begin{frame}[t]{}
\end{frame}

\begin{frame}[t]{}
\end{frame}

\begin{frame}[t]{}
\end{frame}

\begin{frame}[t]{}
\end{frame}

